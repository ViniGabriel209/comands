\documentclass[a4paper,12pt]{article} % ,twocolumn para duas colunas
\usepackage[utf8]{inputenc} % Acentos
\usepackage[lmargin=3cm,tmargin=3cm,rmargin=2cm,bmargin=2cm]{geometry} % Margens
\usepackage[x]{setspace} % Espaçamento (singlespacing, onehalfspacing ou doublespacing)
\usepackage[T1]{fontenc}
\usepackage[brazil]{babel} % Pacote em português
\usepackage{graphicx,xcolor,comment,enumerate,multirow,multicol} 
\usepackage{indentfirst} % Parágrafo na primeira linha 
\usepackage{amsmath,amsthm,amsfonts,amssymb,dsfont,mathtools,blindtext}
\usepackage{multicol} % Para botar o texto em colunas
\usepackage{caption} % Para colocar legenda de autoria 

\usepackage{times}  % Fonte times

\usepackage{helvet}
\renewcommand{\familydefault}{\sfdefault}  % Fonte arial

\usepackage{lmodern} % Corrige o problema das letras estranhas
\usepackage{mathptmx} % Outra alternativa para fonte, parece com times, testar no beamer
% OUTRAS ALTERNATIVAS, RECOMENDO AS DE ARIAL OU TIMES

\setlength{\parindent}{2cm} % Tamanho da indentação do parágrafo (2 cm por exemplo)
\setlength{\parskip}{0.2cm} % Tamanho do espaço entre os parágrafos (0,2 cm por exemplo) (opcional)

\title{}
\author{}
\date{}
\pagestyle{empty} % RETIRAR A NUMERAÇÃO DAS PÁGINAS
% -------------------------------------------------------
% -------------------------------------------------------


\begin{document}

\begin{multicols}{x} % x É O NÚMERO DE COLUNAS PARA O TEXTO . USAR O PACOTE {multicol}
TEXTO
\end{multicols}
% BOTAR multicols* PARA FAZER O TEXTO DESEQUILIBRADO, ANTES AS COLUNAS FICAVAM COM A MESMA PROFUNDIDADE   RECOMENDADO

\begin{titlepage}
	\maketitle
	Outras coisas
\end{titlepage}
% CAPA

\begin{abstract}
\end{abstract}
% RESUMO DO ARTIGO

\begin{flushleft}
\end{flushleft}

\begin{flushright}
\end{flushright}

\begin{center}
\end{center}
% ALINHAMENTO DO TEXTO    OLHAR EM LaTeX

\begin{quote}
\end{quote}
% USADO PARA FAZER CITAÇÕES

\begin{quotation}
\end{quotation}
% USADO PARA CITAÇÕES COM PARÁGRAFO

\begin{enumerate}
\end{enumerate}
% FAZ LISTA COM NÚMEROS (USAR \item) (PODE USAR ENUMERATE DENTRO DE OUTRO ENUMERATE) 

\begin{itemize}
\end{itemize}
% FAZ LISTA COM PONTINHOS 

\begin{equation}
\end{equation}
% FAZ UMA EQUAÇÃO CENTRALIZADA E NUMERADA. USAR \label{} e \ref{}

\usepackage{lscape}
\begin{landscape}
\end{landscape}
% SERÁ CRIADA UMA NOVA PÁGINA EM MODO PAISAGEM. ÚTIL PARA TABELAS MUITO GRANDES

\hfill   % PÕE A MINIPAGE NA MARGEM DIREITA
\begin{minipage}{xcm}
\end{minipage}
% PÕE O TEXTO COM A LARGURA DE x cm, INTERESSANTE PARA FAZER CITAÇÕES NA ABNT

\parbox{width}{texto} 
% FAZ UM TEXTO COM A LARGURA DEFINIDA, INTERESSANTE QUANDO SE DESEJA FAZER QUEBRA DE LINHA

\textsuperscript{} / \textsubscript{}  % PARA COLOCAR TEXTO NO EXPOENTE OU NO ÍNDICE. ex n°
\textrm{}  % PARA ESCREVER TEXTO EM AMBIENTE MATEMÁTICO

\section{}
% PARA FAZER SEÇÃO (PARA TIRAR NUMERAÇÃO BOTAR \section*{} )
\appendix 
% ANTES DA SEÇÃO PARA TORNAR APÊNCICE

\\[x*cm]
% PULAR LINHA *[OPCIONAL] 

%%%%%%%%%%%%%%
\newcommand{\PR}[1]{\ensuremath{\left[#1\right]}}
\newcommand{\PC}[1]{\ensuremath{\left(#1\right)}}
\newcommand{\chav}[1]{\ensuremath{\left\{#1\right\}}}

\PC \PR \chav
% Parênteses curvos, retos e chavetas já com o tamanho correto

\tableofcontents
% SUMÁRIO (COMPILAR TRÊS VEZES)
\renewcommand{\contentsname}{Índice}
% PARA MUDAR O NOME DO SUMÁRIO (Índice por exemplo)

\setlength{\parindent}{2cm}
% TAMANHO DA INDENTAÇÃO DO PARÁGRAFO (2 cm por exemplo)

\newpage
% NOVA PÁGINA

\hspace{1cm} ou \vspace{1cm}
% DÁ ESPAÇAMENTO ENTRE PALAVRAS OU LINHAS (1 cm por exemplo)

\noindent
% PARA TIRAR A INDENTAÇÃO DE PARÁGRAFO
 
\newcommand{\comando}{O que quer fazer}
% POR EXEMPLO ESCREVER UM SINÔNIMO PARA ALGO GRANDE USADO VÁRIAS VEZES

\footnote{Nota}
% PARA FAZER UMA NOTA DE RODAPÉ. SERÁ GERADO A NUMERAÇÃO AUTOMATICAMENTE
% VER A APOSTILA INTERMEDIÁRIO PARA MAIS INFORMAÇÕES SOBRE RODAPÉ E CABEÇALHO

\href{link}{x} 
% PARA COLOCAR UM LINK, x=NOME QUE IRÁ APARECER NO TEXTO PARA CLICAR
\textcolor{cor}{texto}
% PARA MUDAR A COR DO TEXTO, INTERESSANTE EM LINKS O AZUL 

% -------------------------------------------------------
% -------------------------------------------------------
\listoftables % PARA FAZER A LISTA DE TABELAS
\begin{table}[!x]  % x=h PARA DEIXAR ONDE VOCÊ DESEJA  x=t NO TOPO  x=b NA BASE  ! PARA FORÇAR
   \centering  % PARA CENTRALIZAR A TABELA
   \caption{nome da tabela}  % PARA DAR NOME À TABELA
   \begin{tabular}{l{}p{}c{}r{}}
      \multicolumn{2}{c}{texto} & \multicolumn{2}{c}{texto} \\  % USADO PARA MESCLAR COLUNAS 
      alguma coisa & alguma coisa & alguma coisa & alguma coisa \\
      \hline % SE QUISER LINHA HORIZONTAL
      outra coisa  & outra coisa  & \multicolumn{2}{c}{texto que preenche 2 colunas} \\ 
      \cline{i-j} % PARA TRAÇAR LINHA HORIZONTAL DA COLUNA i À COLUNA j
   \end{tabular}
   % O NÚMERO DE LETRAS NO {} É O NÚMERO DE COLUNAS, BOTAR DENTRO DE {xcm} O TAMANHO DA COLUNA 
   % l- ALINHADO À ESQUERDA  c- ALINHADO NO CENTRO r- ALINHADO À DIREITA
   % SEPARAR CADA COLUNA COLOCANDO &, E NO FINAL \\ PARA IR PARA A OUTRA LINHA
   % BOTAR NAS LETRAS DAS COLUNAS BARRAS VERTICAIS (|) PARA LINHAS VERTICAIS NA TABELA
   \caption*{escrever a fonte}  % PARA DAR AUTORIA À TABELA
   \label{nome para referência} % DEPOIS \ref{nome para referência} PARA MOSTRAR NO TEXTO
\end{table}
\usepackage{booktabs} 
\addlinespace[xpt] % DENTRO DO TABULAR PARA ADICIONAR ESPAÇO ENTRE AS LINHAS. USAR O PACOTE
\toprule  \midrule  \bottomrule
% PARA A LINHA DE CIMA, ALGUMA DO MEIO E A DE BAIXO. FICA MAIS BONITO
\cmidrule{i-j} % PARA SUBSTITUIR \cline{i-j}
% -------------------------------------------------------
% -------------------------------------------------------


% -------------------------------------------------------
% -------------------------------------------------------
\listoffigures % PARA FAZER LISTA DE FIGURAS
\begin{figure}[!x]  % x=h PARA DEIXAR ONDE VOCÊ DESEJA  x=t NO TOPO  x=b NA BASE  ! PARA FORÇAR
   \centering % PARA CENTRALIZAR A FIGURA
   \caption{nome da figura a ser dado} % PARA DAR NOME À FIGURA
   \includegraphics[scale=•,angle=]{NOME DA PASTA/NOME DA FIGURA NA PASTA JUNTO DO ARQUIVO DO TEXTO}
   \caption*{escrever a fonte}  % PARA DAR AUTORIA À FIGURA
   \label{nome para referência} % DEPOIS \ref{nome para referência} PARA MOSTRAR NO TEXTO
\end{figure}
% -------------------------------------------------------
% -------------------------------------------------------


\cite{apelido} % EM ALGUM LUGAR DO TEXTO
\bibliographystyle{x} % USAR ieeetr
\usepackage[alf]{abntex2cite} % FAZ AS CITAÇÕES NA ABNT (NOME,DATA), ALTERNATIVA AO COMANDO ANTERIOR
\addcontentsline{toc}{section}{nome no índice} % OPCIONAL PARA BOTAR NO SUMÁRIO
\bibliography{nomedoarquivo.bib} % CRIAR OUTRO ARQUIVO COM AS REFERENCIAS, NO FINAL .bib
% IR EM Ferramentas / BibTex PARA COMPILAR. COMPILAR MAIS VEZES

% COMO FAZER A REFERÊNCIA NO ARQUIVO .bib NA MESMA PASTA DO ARQUIVO PRINCIPAL
% NO GOOGLE SCHOLAR, QUANDO UTILIZAR REFERÊNCIA DE ALGUM ARTIGO, NA SUA ABA ANTES DE ENTRAR NO SITE HAVERÁ ASPAS, CLIQUE NELA E VAI APARECER BibTex. COPIE, COLE E SEJA FELIZ
@BOOK{Apelido, 
author={N1. Sobrenome1 and N2. Sobrenome2},
title={Nome do Livro},
publisher={Editora},
edition={ \textsuperscript{o}},
address={},
year={ano},
volume={volume}
}
@article{Apelido,
author={N. Sobrenome1 and N. Sobrenome2},
title={Nome do artigo},
journal={Nome do peri\'odico},
year={ano},
volume={volume},
number={n\'umero},
pages={PaginaInicial-PaginaFinal},
}
@inproceedings{Apelido,
author={N. Sobrenome1 and N. Sobrenome2},
title={Nome do artigo},
booktitle={Nome dos anais do congresso},
year={ano},
pages={PaginaInicial-PaginaFinal},
address={cidade, pa\'is},
}
@misc{Apelido,
Author={Lauro C{\'e}sar Araujo},
Date-Added={2013-03-23 21:39:21 +0000},
Date-Modified={2015-04-27 22:43:06 +0000},
Howpublished={Wiki do abnTeX2},
Keywords={wiki},Title={Como customizar o abnTeX2},
Url={https://github.com/abntex/abntex2/wiki/ComoCustomizar},
Urlaccessdate={27 abr 2015},
Year={2015},
Bdsk-Url-1={https://github.com/abntex/abntex2/wiki/ComoCustomizar}
}

% TAMANHOS DAS FONTES
                    10pt    11pt    12pt
\tiny               5       6       6
\scriptsize         7       8       8
\footnotesize       8       9       10
\small              9       10      10.95
\normalsize         10      10.95   12
\large              12      12      14.4
\Large              14.4    14.4    17.28
\LARGE              17.28   17.28   20.74
\huge               20.74   20.74   24.88
\Huge               24.88   24.88   24.88

\end{document}

%%%%%%%%%%%%%%%%%%%%%%%%%%%%%
%% APRESENTAÇÃO COM BEAMER %%
%%%%%%%%%%%%%%%%%%%%%%%%%%%%%

\documentclass[x]{beamer} % x=aspectratio=169 (OPCIONAL) PARA ELIMINAR AS BARRAS PRETAS NA LATERAL
\usepackage[utf8]{inputenc} % Acentos
\usetheme{} % ESCOLHER O TEMA DA APRESENTAÇÃO (OPCIONAL) PROCURAR EM Assistentes -> QB Presentation

\title{}
\subtitle{}
\author{}       % USAR \and OU \\ PARA MAIS AUTORES (TESTAR)
\institute{}
\date{}         % DEIXAR VAZIO PARA FICAR SEM DATA, USAR TAMBÉM PARA COLOCAR O LUGAR ex: lugar, ano
% INFORMAÇÕES RELEVANTES PARA COLOCAR NO PRIMEIRO QUADRO
% [VERSÃO CURTA] ANTES DO {} PARA CORTE NO RODAPÉ

\logo{\includegraphics[scale=•]{NOME DA PASTA/NOME DA IMAGEM}} 
% PARA COLOCAR UMA IMAGEM QUE APARECERÁ EM TODOS OS QUADROS (TESTAR PQ PODE FICAR ESTRANHO)(OPCIONAL)

\begin{document}

\begin{frame}
\frametitle{Título}          % TÍTULO DE CADA QUADRO (OPCIONAL)  FICA NO TOPO
\framesubtitle{Subtítulo}    % SUBTÍTULO DE CADA SLIDE     
% CONTEÚDO
\end{frame}
% CADA QUADRO SE DÁ DENTRO DESTE AMBIENTE, UM QUADRO PODE TER MAIS DE UM SLIDE 

\titlepage  
% PARA COLOCAR NO PRIMEIRO SLIDE, DENTRO DE UM FRAME

\alert{texto em destaque}  
% DESTACA O TEXTO, PÕE EM VERMELHO POR EXEMPLO

\pause
% LOGO APÓS ALGUM COMANDO DENTRO DO FRAME (MOTIVOS ÓBVIOS)

\setbeamercover{transparent} % NO PREÂMBULO
\uncover<x-y>{Texto} % CRIA VÁRIOS SLIDES DENTRO DE UM QUADRO, CADA "ITEM" FICARÁ DESTACADO E OS OUTROS TRANSPARENTES, ELE APARECE DO SLIDE x AO y(-y OPCIONAL), <x-> APARECE ATÉ O FINAL
% PROCURAR POR \visible E \only (NÃO USO)  % PARECIDO COM O \pause

% NOS AMBIENTES A SEGUIR APARECERÃO <x-y> COM A MESMA LÓGICA DO \uncover (TODOS OPCIONAIS)

\begin{block}<x-y>{x} %x={O TÍTULO DO BLOCO}, OU {} OU {\ } OS DOIS ÚLTIMOS PRODUZEM EFEITOS DIFERENTES 
Texto 
\end{block}
% COMEÇA UM BLOCO DE TEXTO, DENTRO DE UMA CAIXINHA, COM OU SEM TÍTULO (DEPENDE DO {})

\begin{itemize}
\item<x-y | ação@w-z> Texto  % ITEM APARECERÁ NOS SLIDES x-y (COMO \uncover)
% ação=(alert, uncover, visible, only) AÇÃO ACONTECE NOS SLIDES w-z	
\end{itemize}
% FAZ LISTA COM PONTOS (TÓPICOS DO QUADRO)

\section[x]{}         % COLOCAR ANTES DO FRAME, x=título curto (OPCIONAL)
\subsection[]{}       % VALE O MESMO, OUTRAS subsubsections SÃO POSSÍVEIS

\tableofcontents[x]   % SUMÁRIO APÓS PRIMEIRO QUADRO, DENTRO DE UM FRAME, COLOCAR TÍTULO (SUMÁRIO)
                      % x=pausesections, pausesubsections PARA APRESENTAR PAUSADAMENTE AS SEÇÕES

\AtBeginSection[]{
	\begin{frame}
	\tableofcontents[currentsection, hideothersubsections]
	\end{frame}
}
% ANTES DO INÍCIO DE CADA SEÇÃO SERÁ MOSTRADA NOVAMENTE O SUMÁRIO COM A RESPECTIVA SEÇÃO DESTACADA
% ÚTIL PARA GRANDES APRESENTAÇÕES
% POSSÍVEL FAZER COM SUBSEÇÕES (AtBeginSubsection, currentsubsection) 

% REFERÊNCIAS PODEM SER FEITAS DA MESMA FORMA (ARTIGO) DENTRO DE UM FRAME 
% PROCURAR POR allowframebreaks PARA REFERÊNCIAS EM DOIS QUADROS

\end{document}
